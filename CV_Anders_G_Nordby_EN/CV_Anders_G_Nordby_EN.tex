%----------------------------------------------------------------------------------------
% Freeman Curriculum Vitae
% XeLaTeX Template
% Version 2.0 (19/3/2018)
%
% This template originates from:
% http://www.LaTeXTemplates.com
%
% Authors:
% Vel (vel@LaTeXTemplates.com)
% Alessandro Plasmati
%
% License:
% CC BY-NC-SA 3.0 (http://creativecommons.org/licenses/by-nc-sa/3.0/)
%
%!TEX program = xelatex
% NOTICE: This template must be compiled with XeLaTeX, the line above should
% ensure this happens automatically but if it doesn't you will need to specify 
% XeLaTeX as the engine in your editor or script
% 
%----------------------------------------------------------------------------------------


%----------------------------------------------------------------------------------------
%	PACKAGES AND OTHER DOCUMENT CONFIGURATIONS
%----------------------------------------------------------------------------------------

\documentclass[10pt]{article} % Font size, can be: 10pt, 11pt or 12pt

\input{structure.tex} % Include the file that specifies the document structure

% Headers and footers can be added with the \lhead{} \rhead{} \lfoot{} \rfoot{} commands
% Example right footer:
%\rfoot{\color{headings}{\sffamily Last update: \today. Typeset with Xe\LaTeX}}

%----------------------------------------------------------------------------------------

\begin{document}

\begin{paracol}{2} % Begin the multi-column environment

%----------------------------------------------------------------------------------------
%	NAME AND CURRICULUM VITAE TEXT
%----------------------------------------------------------------------------------------

\parbox[top][0.12\textheight][c]{\linewidth}{ % Parbox to hold the author name and CV text; fixed height to match the coloured box to the right, centred vertically and full line width
	\vspace{-0.04\textheight} % Reduce whitespace above the parbox to separate it from the main content
	\centering % Centre text
	{\sffamily\Huge Anders G. Nordby}\\\medskip % Your name
	{\Huge\color{headings}\cvtextfont Curriculum Vitae}
}

%----------------------------------------------------------------------------------------
%	DESCRIPTION
%----------------------------------------------------------------------------------------

\section{Description}

Anders is a project manager, systems architect and senior developer with customer focus and many years of
experience from the industry. He holds the \textbf{PRINCE2 Practitioner}, \textbf{PRINCE2 Foundation} and
\textbf{Certified Scrum Master} certifications, and he has had the role of \textit{scrum master} in several projects.
\medskip

He also has experience as a teacher, which is useful when technical challenges are to be explained 
in a non-technical and jargon-free manner. He has also been mentor for new employees.
\medskip

%----------------------------------------------------------------------------------------
%	RELEVANT PROJECT EXPERIENCE
%----------------------------------------------------------------------------------------

\section{Relevant Project Experience}

\workposition{08.2019 -- 12.2019} 
{Delichon} % FT/PT (full time or part time)
{ABB: Process Power Simulator} 
{Senior Developer} 
{The ABB Process Power Simulator used to simulate larger systems. It has tools for building, configuration and running models. 
It can also operate the models as in a local field environment by interacting with the equipment and create scenarios.

\qquad Some of the tools and technologies Anders used in this project were: \textit{Azure DevOps, Akka.NET, TypeScript, React JS, Microsoft .NET Core 2.2, ASP.NET Core MVC, Keycloak, C\#, Microsoft SQL Server}.} 



\workposition{04.2018 -- 08.2018} 
{Sopra Steria} % FT/PT (full time or part time)
{Norwegian Tax Administration: Intranet} 
{Senior Developer / Architect} 
{Anders worked with maintenance, development and documentation on an existing intranet solution called "the SKO portal". The solution was based on \textit{Episerver CMS}. Anders contributed with training for the internal employees, and he created a set of recipes for simplifying the editors' daily work.

\qquad Some of the tools and technologies Anders used in this project were: \textit{Episerver CMS, Microsoft .Net, C\#, Microsoft SQL Server, Ajax, Javascript, Jira, Confluence, Git, Octopus Deploy}.} 

\workposition{05.2016 -- 08.2017} 
{CGI} % FT/PT (full time or part time)
{ASKO: Better Selling Commerce Website} 
{Senior Developer} 
{ASKO is Norway's largest grocery wholesale, and the project aimed at creating a new \textsc{b2b} portal that would be both more user friendly and also selling more products through better recommendations.

\qquad Anders was responsible for installing the \textit{inRiver PIM} system into the solution, optimizing it and develop integrations between the \textit{inRiver PIM} system and the customer's various internal systems. He was also responsible for performance testing of
the solution using \textit{jMeter} and \textit{Blazemeter}. Anders developed an authentication system based on \textit{IdentityServer3}
for handling logins.

\qquad Some of the tools and technologies Anders used in this project were: \textit{Episerver CMS, Episerver Commerce, inRiver PIM, Microsoft .Net, C\#, Microsoft SQL Server, REST Services, PowerShell, WebAPI, WCF, IdentityServer3, Jira, Confluence, TFS, Git, Octopus Deploy}.} 
\pagebreak


\workposition{05.2015 -- 12.205} 
{CGI} 
{Fellesforbundet: New Public Website} 
{Senior Developer, Team Lead, Scrum Master, Mentor} 
{Fellesforbundet is one of Norway's largest trade unions.

\qquad Anders was responsible for planning and estimation of the solution, and also for coordinating the work between the developers on the team, and for keeping the customer updated about the progress. Anders participated hands-on in development of the solution, which was based on textit{Episerver CMS}.

\qquad One of the developers came straight from college, and Anders was assigned the role as mentor, teaching the new employee about \textit{Episerver CMS} and also about the various internal systems at CGI. 

\qquad Some of the tools and technologies Anders used in this project were: \textit{Episerver CMS, Microsoft .Net, C\#, Microsoft SQL Server, REST Services, PowerShell, WebAPI, Jira, Confluence, TFS, Octopus Deploy}.} 

\workposition{03.2014 -- 05.2015} 
{CGI} 
{Tekna: New Public Website} 
{Senior Developer, Team Lead} 
{Tekna is a trade union for technical and scientific professionals.

\qquad The project consisted of two sub projects, a CMS-project and a CRM-project. Anders was responsible for planning, estimation and coordination among the CMS developers. As a developer, Anders was responsible for the integration between the CMS and CRM projects.

\qquad Some of the tools and technologies Anders used in this project were: \textit{Episerver CMS, Microsoft .Net, C\#, Microsoft SQL Server, REST Services, PowerShell, WebAPI, Jira, Confluence, TFS, Octopus Deploy, NServiceBus, Knockout.js, Javascript, jQuery}.} 

\workposition{08.2012 -- 03.2014} 
{Making Waves} 
{Stortinget: New Intranet} 
{Senior Developer, Mentor} 
{Anders was responsible for creating a solution gathering calendar events from some 30-odd calendars, letting end-users find their calendar information in a single place. He also was responsible for integrating \textit{Episerver CMS} with the \textit{IntelliSearch} search engine, and also for the integration between the new \textit{Episerver CMS} solution and the \textit{Sharepoint} solution that was developed in parallel.

\qquad During the project, Anders also fulfilled the role as mentor for two new employees.

\qquad Some of the tools and technologies Anders used in this project were: \textit{Episerver CMS, Microsoft .Net, C\#, Microsoft SQL Server, REST Services, PowerShell, IntelliSearch, TFS, Octopus Deploy, Making Waves ActivityFeed, Knockout.js, Javascript, jQuery, Jira, Confluence}.} 



\workposition{02.2010 -- 03.2011} 
{Tarantell} 
{Synoptik: New public websites for Brilleland \& Interoptik} 
{Senior Developer, Team Lead, Scrum Master} 
{Anders was responsible for the planning and development of two new web solutions based on \textit{Episerver CMS}, but with as much code re-use as possible to keep the costs down.

\qquad Some of the tools and technologies Anders used in this project were: \textit{Episerver CMS, Microsoft .Net, C\#, Microsoft SQL Server, PowerShell, TFS, Octopus Deploy, Javascript, jQuery, FlusterMaps, Google Mini, Jira, Confluence}.} 

%------------------------------------------------

\vspace{-\baselineskip}\medskip % Standardise the whitespace after this section and before the next (the custom command adds too much otherwise)

\switchcolumn % Switch to the next paracol column

%----------------------------------------------------------------------------------------
%	COLOURED CONTACT DETAILS BOX
%----------------------------------------------------------------------------------------

\parbox[top][0.12\textheight][c]{\linewidth}{ % Parbox to hold the colour box; fixed height to match the name/CV text to the left, centred vertically and full line width
	\vspace{-0.04\textheight} % Reduce whitespace above the parbox to separate it from the main content
	\colorbox{shade}{ % Create the coloured box
		\begin{supertabular}{p{0.05\linewidth}|p{0.775\linewidth}} % Start a table with two columns, the table will ensure everything is aligned
			\raisebox{-1pt}{\faHome} & Motzfeldtsgate 14, N-0187 Oslo, NORWAY \\ % Address
			\raisebox{-1pt}{\faPhone} & (+47) 932 20 333 \\ % Phone number
			\raisebox{0pt}{\small\faEnvelope} & \href{mailto:anders@delichon.no}{anders@delichon.no} \\ % Email address
%			\raisebox{0pt}{\small\faEnvelope} & \href{mailto:anders.nordby@gmail.com}{anders.nordby@gmail.com} \\ % Email address
			\raisebox{-1pt}{\small\faDesktop} & \href{http://delichon.no/services/anders/}{http://delichon.no/services/anders/} \\ % Website
			%\raisebox{-1pt}{\faGithub} & \href{https://github.com/username}{https://github.com/username} \\ % GitHub profile
			\raisebox{-1pt}{\faLinkedinSquare} & \href{https://www.linkedin.com/in/agnordby/}{https://www.linkedin.com/in/agnordby/} \\ % LinkedIn profile
			% See fontawesome.pdf in the fonts folder for all icons you can use
		\end{supertabular}
	}
}

%----------------------------------------------------------------------------------------
%	KEY COMPETENCIES
%----------------------------------------------------------------------------------------

\section{Key Competencies}

\longformdescription{Processes}{Systems Architecture, Scrum, Kanban, Project Management.}

\longformdescription{Development}{Episerver CMS, Episerver Commerce, Episerver Find, C\#, ASP.Net, MVC, Apptus eSales, inRiver PIM, IdentityServer, T-SQL, PL/SQL, JavaScript, JSON, Ajax, NServiceBus, WCF, WebServices, REST, WebAPI, XML, HTML, CSS, jQuery, PowerShell, \LaTeX.}

\longformdescription{Tools}{Jira, Confluence, Git, TFS, SubVersion, ReSharper, NewRelic, jMeter, BlazeMeter, NuGet.}


%----------------------------------------------------------------------------------------
%	CERTIFICATIONS
%----------------------------------------------------------------------------------------

\section{Certifications}

% Example \tableentry{} command to add another line:

%\tableentry{Heading}{Content}{spaceafter}

% All 3 parameters must be supplied but any can be empty if you don't need them
% A "spaceafter" value in the third parameter will add some vertical space -- this is to be used between headings

%------------------------------------------------

\begin{supertabular}{rl} % Start a table with two columns, the table will ensure everything is aligned

	\tableentry{2019}{\textbf{PRINCE2 Practitioner}}{}
	\tableentry{}{\textit{PeopleCert}}{spaceafter}
	
	\tableentry{2019}{\textbf{Certified Scrum Master}}{}
	\tableentry{}{\textit{Scrum Alliance}}{spaceafter}

	\tableentry{2018}{\textbf{PRINCE2 Foundation}}{}
	\tableentry{}{\textit{PeopleCert}}{spaceafter}

	\tableentry{2018}{\textbf{Programming HTML5 with Javascript \& CSS3}}{}
	\tableentry{}{\textit{Microsoft}}{spaceafter}

	\tableentry{2017}{\textbf{Certified Episerver Developer}}{}
	\tableentry{}{\textit{Episerver}}{spaceafter}

	\tableentry{2016}{\textbf{Certified inRiver Developer}}{}
	\tableentry{}{\textit{inRiver}}{spaceafter}



\end{supertabular}



%----------------------------------------------------------------------------------------
%	COURSES
%----------------------------------------------------------------------------------------

\section{Courses}

% Example \tableentry{} command to add another line:

%\tableentry{Heading}{Content}{spaceafter}

% All 3 parameters must be supplied but any can be empty if you don't need them
% A "spaceafter" value in the third parameter will add some vertical space -- this is to be used between headings

%------------------------------------------------

\begin{supertabular}{rl} % Start a table with two columns, the table will ensure everything is aligned

	\tableentry{2019}{\textbf{PRINCE2 Practitioner}}{}
	\tableentry{}{\textit{Metier OEC}}{spaceafter}

	\tableentry{2019}{\textbf{Certified Scrum Master Course}}{}
	\tableentry{}{\textit{Glasspaper}}{spaceafter}

	\tableentry{2019}{\textbf{Architecture Engineering Workshop}}{}
	\tableentry{}{\textit{Tom Gilb}}{spaceafter}


	
	\tableentry{2018}{\textbf{Software Architechture School}}{}
	\tableentry{}{\textit{Sopra Steria}}{spaceafter}




	\tableentry{2016}{\textbf{inRiver Certified Developer}}{}
	\tableentry{}{\textit{inRiver}}{spaceafter}

	\tableentry{2015}{\textbf{Lead Enterprise Architect Programme (LEAP)}}{}
	\tableentry{}{\textit{Microsoft}}{spaceafter}
	
\end{supertabular}




%----------------------------------------------------------------------------------------
%	EDUCATION
%----------------------------------------------------------------------------------------

\section{Education} 

% Blank \educationentry{} command to add another degree:

%\educationentry{} 
%{} % Degree
%{} % Honours, achievements or distinctions (e.g. first class honours)
%{} % Department
%{} % Institution

% All 5 parameters must be supplied but any can be empty if you don't need them

%------------------------------------------------

\begin{supertabular}{rl} % Start a table with two columns, the table will ensure everything is aligned

	%------------------------------------------------
	
	\educationentry{2003 -- 2004} 
	{Secondary Education Teacher Training} % Degree
	{Bachelor's Degree Additional Courses} % Honours, achievements or distinctions (e.g. first class honours)
	{} % Department
	{University of Oslo} % Institution
	
	\educationentry{1999 -- 2000} 
	{IT Candidate} % Degree
	{} % Honours, achievements or distinctions (e.g. first class honours)
	{Object Oriented Programming.} % Department
	{The Business Academy, Oslo} % Institution
	
	\educationentry{1987 -- 1996} 
	{Candidatus Magisterii} % Degree
	{Old grade between Bachelor and Master} % Honours, achievements or distinctions (e.g. first class honours)
	{Mathematics, Computer Science, Japanese.} % Department
	{University of Oslo} % Institution


	%------------------------------------------------

\end{supertabular}


%----------------------------------------------------------------------------------------
%	EMPLOYMENT HISTORY
%----------------------------------------------------------------------------------------

\section{Employment History} 

% Blank \educationentry{} command to add another degree:

%\educationentry{} 
%{} % Degree
%{} % Honours, achievements or distinctions (e.g. first class honours)
%{} % Department
%{} % Institution

% All 5 parameters must be supplied but any can be empty if you don't need them

%------------------------------------------------

\begin{supertabular}{rl} % Start a table with two columns, the table will ensure everything is aligned

	\educationentry{05.2019 -- present} 
	{Systems Architect \& Senior Developer} % Degree
	{} % Honours, achievements or distinctions (e.g. first class honours)
	{} % Department
	{Delichon} % Institution

	\educationentry{05.2019 -- present} 
	{Daily Manager} % Degree
	{} % Honours, achievements or distinctions (e.g. first class honours)
	{} % Department
	{Delichon} % Institution

	\educationentry{01.2018 -- 04.2019} 
	{Lead Software Engineer} % Degree
	{} % Honours, achievements or distinctions (e.g. first class honours)
	{} % Department
	{Sopra Steria} % Institution
	
	
	\educationentry{03.2014 -- 08.2017} 
	{Senior Web Developer} % Degree
	{} % Honours, achievements or distinctions (e.g. first class honours)
	{} % Department
	{CGI} % Institution
	

	\educationentry{03.2010 -- 02.2014} 
	{Senior Systems Consultant} % Degree
	{Tarantell merged with Making Waves} % Honours, achievements or distinctions (e.g. first class honours)
	{} % Department
	{Making Waves / Tarantell} % Institution
	
	\educationentry{08.2008 -- 12.2009} 
	{Consultant / Systems Developer} % Degree
	{} % Honours, achievements or distinctions (e.g. first class honours)
	{} % Department
	{Creuna} % Institution
	

	\educationentry{02.2006 -- 07.2008} 
	{Systems Developer \& Project Manager} % Degree
	{InfoFinder bought Ocelluz} % Honours, achievements or distinctions (e.g. first class honours)
	{} % Department
	{InfoFinder / Ocelluz} % Institution
	
	
	\educationentry{10.2004 -- 02.2008} 
	{Self Employed} % Degree
	{} % Honours, achievements or distinctions (e.g. first class honours)
	{} % Department
	{Anders Nordby Konsulenttjenester} % Institution
	
	\educationentry{05.1995 -- 06.2002} 
	{Consultant} % Degree
	{} % Honours, achievements or distinctions (e.g. first class honours)
	{} % Department
	{Ajilon} % Institution
	
	\educationentry{08.1994 -- 07.1995} 
	{Substitute Teacher} % Degree
	{} % Honours, achievements or distinctions (e.g. first class honours)
	{} % Department
	{Gran School, Oslo} % Institution
	

\end{supertabular}


%----------------------------------------------------------------------------------------
%	AWARDS
%----------------------------------------------------------------------------------------

\section{Awards}

% Example \tableentry{} command to add another line:

%\tableentry{Heading}{Content}{spaceafter}

% All 3 parameters must be supplied but any can be empty if you don't need them
% A "spaceafter" value in the third parameter will add some vertical space -- this is to be used between headings

%------------------------------------------------

\begin{supertabular}{rl} % Start a table with two columns, the table will ensure everything is aligned
	
	\tableentry{2013}{\textbf{Episerver Most Valued Professional (EMVP)}}{}
	\tableentry{}{\textit{Expired 2017}}{spaceafter}
	
\end{supertabular}


%----------------------------------------------------------------------------------------
%	LANGUAGES
%----------------------------------------------------------------------------------------

\section{Languages} 

% Example \tableentry{} command to add another line:

%\tableentry{Heading}{Content}{spaceafter}

% All 3 parameters must be supplied but any can be empty if you don't need them
% A "spaceafter" value in the third parameter will add some vertical space -- this is to be used between headings

%------------------------------------------------

\begin{supertabular}{rl} % Start a table with two columns, the table will ensure everything is aligned
	
	%------------------------------------------------
	
	\tableentry{Excellent}{Norwegian (native), English}{spaceafter}
	\tableentry{Intermediate}{Bulgarian}{spaceafter}
	\tableentry{Beginner}{Spanish, German, French, Japanese}{spaceafter}
	\tableentry{For fun}{High Valyrian}{spaceafter}
	
	%------------------------------------------------

\end{supertabular}


%----------------------------------------------------------------------------------------

\end{paracol}

%----------------------------------------------------------------------------------------

\end{document}
