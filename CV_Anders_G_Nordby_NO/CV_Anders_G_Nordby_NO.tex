%----------------------------------------------------------------------------------------
% Freeman Curriculum Vitae
% XeLaTeX Template
% Version 2.0 (19/3/2018)
%
% This template originates from:
% http://www.LaTeXTemplates.com
%
% Authors:
% Vel (vel@LaTeXTemplates.com)
% Alessandro Plasmati
%
% License:
% CC BY-NC-SA 3.0 (http://creativecommons.org/licenses/by-nc-sa/3.0/)
%
%!TEX program = xelatex
% NOTICE: This template must be compiled with XeLaTeX, the line above should
% ensure this happens automatically but if it doesn't you will need to specify 
% XeLaTeX as the engine in your editor or script
% 
%----------------------------------------------------------------------------------------


%----------------------------------------------------------------------------------------
%	PACKAGES AND OTHER DOCUMENT CONFIGURATIONS
%----------------------------------------------------------------------------------------

\documentclass[10pt]{article} % Font size, can be: 10pt, 11pt or 12pt

\input{structure.tex} % Include the file that specifies the document structure

% Headers and footers can be added with the \lhead{} \rhead{} \lfoot{} \rfoot{} commands
% Example right footer:
%\rfoot{\color{headings}{\sffamily Last update: \today. Typeset with Xe\LaTeX}}

%----------------------------------------------------------------------------------------

\begin{document}

\begin{paracol}{2} % Begin the multi-column environment

%----------------------------------------------------------------------------------------
%	NAME AND CURRICULUM VITAE TEXT
%----------------------------------------------------------------------------------------

\parbox[top][0.12\textheight][c]{\linewidth}{ % Parbox to hold the author name and CV text; fixed height to match the coloured box to the right, centred vertically and full line width
	\vspace{-0.04\textheight} % Reduce whitespace above the parbox to separate it from the main content
	\centering % Centre text
	{\sffamily\Huge Anders G. Nordby}\\\medskip % Your name
	{\Huge\color{headings}\cvtextfont Curriculum Vitae}
}

%----------------------------------------------------------------------------------------
%	DESCRIPTION
%----------------------------------------------------------------------------------------

\section{Beskrivelse}

Anders er prosjektleder, systemarkitekt og seniorutvikler med kundefokus og han har mange års erfaring fra branjen.
Han er sertifisert som \textbf{PRINCE2 Practitioner}, \textbf{PRINCE2 Foundation} og \textbf{Certified Scrum Master}, og 
har hatt rollen som \textit{scrum master} i flere prosjekter.
\medskip

Han har også erfaring som lærer, noe som kommer godt med i mange sammenhenger, blant annet når tekniske utfordringer
skal forklares på en ikke-teknisk og lettforståelig måte. Han har også fungert som mentor for nyansette en rekke ganger.
\medskip

%----------------------------------------------------------------------------------------
%	RELEVANT PROJECT EXPERIENCE
%----------------------------------------------------------------------------------------

\section{Relevant Prosjekterfaring}

\workposition{08.2019 -- 12-2019} 
{Delichon} % FT/PT (full time or part time)
{ABB: Process Power Simulator} 
{Senior Developer} 
{ABB Process Power Simulator brukes for å simulere store systemer. Systemet inneholder verktøy for å bygge, konfigurere og kjøre modeller. 
Det kan også operere modeller i et lokalt miljø ved å interagere med utstyr, samt lage scenarier.

\qquad Noen verktøy og teknologier Anders brukte i prosjektet: \textit{Azure DevOps, Akka.NET, TypeScript, React JS, Microsoft .NET Core 2.2, ASP.NET Core MVC, Keycloak, C\#, Microsoft SQL Server}.} 


\workposition{04.2018 -- 08.2018} 
{Sopra Steria} % FT/PT (full time or part time)
{Skatteetaten: SKO-porten -- Intranett} 
{Seniorutvikler / arkitekt} 
{Anders jobbet med vedlikehold, utvikling og dokumentasjon av en allerede eksisterende intranettløsning for skatteoppkreverne. Lønsingen var basert på \textit{Episerver CMS}. Anders bidro med opplæring av interne ansatte, og han laget en rekke oppskrifter for å forenkle redaktørenes hverdag.

\qquad Noen verkøy og teknologier Anders brukte i prosjektet: \textit{Episerver CMS, Microsoft .Net, C\#, Microsoft SQL Server, Ajax, Javascript, Jira, Confluence, Git, Octopus Deploy}.} 

\workposition{05.2016 -- 08.2017} 
{CGI} % FT/PT (full time or part time)
{ASKO: MSAN -- Mer Selgende ASKO Netthandel} 
{Seniorutvikler} 
{Målt i antall varelinjer pr dag, er ASKOs netthandel en av landets største nettbutikker. Formålet med prosjektet var å lage en mer brukervennlig nettbutikk, samt å øke omsetningen ved å gi bedre anbefalinger av tilleggsprodukter.

\qquad Anders var ansarlig for installasjon, optimalisering og tilpassing av \textit{inRiver PIM} i løsningen, og for å integrere 
\textit{inRiver PIM} med kundens interne systemer. Han hadde også ansvar for gjennomføre ytelsestesting av løsningen 
med \textit{jMeter} og  \textit{Blazemeter}.

\qquad Anders utviklet en autentiseringsløsning basert på \textit{IdentityServer3}
for å håndtere innlogginger.

\qquad Noen verkøy og teknologier Anders brukte i prosjektet: \textit{Episerver CMS, Episerver Commerce, inRiver PIM, Microsoft .Net, C\#, Microsoft SQL Server, REST Services, PowerShell, WebAPI, WCF, IdentityServer3, Jira, Confluence, TFS, Git, Octopus Deploy}.} 
\pagebreak


\workposition{05.2015 -- 12.205} 
{CGI} 
{Fellesforbundet: Nye websider} 
{Seniorutvikler, teamleder, scrum-master, mentor} 
{Anders var ansvarlig for planlegging og estimering av lønsningen, og også for koordinering av arbeidet mellom utviklerne på teamet. 
Han var også ansvarlig for å holde kunden oppdart om fremdriften. Anders deltok hands-on i utviklingsarbeidet. Løsningen var basert
på \textit{Episerver CMS}.

\qquad En av utviklerne på teamet kom rett fra høyskolen, og Anders ble tildelt rollen som menter, og lærte den nyansatte både om 
\textit{Episerver CMS} og om CGIs mange interne systemer. 

\qquad Noen verkøy og teknologier Anders brukte i prosjektet: \textit{Episerver CMS, Microsoft .Net, C\#, Microsoft SQL Server, REST Services, PowerShell, WebAPI, Jira, Confluence, TFS, Octopus Deploy}.} 

\workposition{03.2014 -- 05.2015} 
{CGI} 
{Tekna: Nytt nettsted} 
{Seniorutvikler, teamleder} 
{Prosjektet besto av to del-prosjekter, et CMS-prosjekt og et CRM-prosjekt. Anders var ansvarlig for planlegging, etstimering og koordinering av teamet som utviklet CMS-prosjektet. Han var også ansvarlig for å utvikle integrasjonen mellom CMS- og CRM-prosjektet.

\qquad Noen verkøy og teknologier Anders brukte i prosjektet: \textit{Episerver CMS, Microsoft .Net, C\#, Microsoft SQL Server, REST Services, PowerShell, WebAPI, Jira, Confluence, TFS, Octopus Deploy, NServiceBus, Knockout.js, Javascript, jQuery}.} 

\workposition{08.2012 -- 03.2014} 
{Making Waves} 
{Stortinget: Nytt intranett} 
{Seniorutvikler, mentor} 
{Anders var ansvarlig for å utvikle en løsning for å samle kalenderinformasjon fra et 30-talls ulike kalendersystemer, slik at
brukerne kunne finne all sin nødvendige kalenderinformasjon på ett sted. Han var også anvarlig for å utvikle integrasjonen mellom 
\textit{Episerver CMS} og søkemotoren \textit{IntelliSearch}, samt for å utvikle integrasjonen mellom \textit{Episerver CMS} og
en \textit{Sharepoint}-løsning som ble utviklet parallelt.

\qquad I prosjektet fikk Anders også rollen som mentor for to nyansatte.

\qquad Noen verkøy og teknologier Anders brukte i prosjektet: \textit{Episerver CMS, Microsoft .Net, C\#, Microsoft SQL Server, REST Services, PowerShell, IntelliSearch, TFS, Octopus Deploy, Making Waves ActivityFeed, Knockout.js, Javascript, jQuery, Jira, Confluence}.} 



\workposition{02.2010 -- 03.2011} 
{Tarantell} 
{Synoptik: Nye websider for Brilleland \& Interoptik} 
{Seniorutvikler, teamleder, scrum-master} 
{Anders var ansvarlig for planlegging og utvikling av to nye web-løsninger basert på \textit{Episerver CMS}, men med så stor grad av gjenbruk av kode i de prosjektene for å holde kostnadene nede.

\qquad Noen verkøy og teknologier Anders brukte i prosjektet: \textit{Episerver CMS, Microsoft .Net, C\#, Microsoft SQL Server, PowerShell, Google Mini, Octopus Deploy, Javascript, jQuery, FlusterMaps, Confluence, TFS, Jira}.} 




%------------------------------------------------

\vspace{-\baselineskip}\medskip % Standardise the whitespace after this section and before the next (the custom command adds too much otherwise)

\switchcolumn % Switch to the next paracol column

%----------------------------------------------------------------------------------------
%	COLOURED CONTACT DETAILS BOX
%----------------------------------------------------------------------------------------

\parbox[top][0.12\textheight][c]{\linewidth}{ % Parbox to hold the colour box; fixed height to match the name/CV text to the left, centred vertically and full line width
	\vspace{-0.04\textheight} % Reduce whitespace above the parbox to separate it from the main content
	\colorbox{shade}{ % Create the coloured box
		\begin{supertabular}{p{0.05\linewidth}|p{0.775\linewidth}} % Start a table with two columns, the table will ensure everything is aligned
			\raisebox{-1pt}{\faHome} & Motzfeldtsgate 14, N-0187 Oslo, NORWAY \\ % Address
			\raisebox{-1pt}{\faPhone} & (+47) 932 20 333 \\ % Phone number
			\raisebox{0pt}{\small\faEnvelope} & \href{mailto:anders@delichon.no}{anders@delichon.no} \\ % Email address
%			\raisebox{0pt}{\small\faEnvelope} & \href{mailto:anders.nordby@gmail.com}{anders.nordby@gmail.com} \\ % Email address
			\raisebox{-1pt}{\small\faDesktop} & \href{http://delichon.no/services/anders/}{http://delichon.no/services/anders/} \\ % Website
			%\raisebox{-1pt}{\faGithub} & \href{https://github.com/username}{https://github.com/username} \\ % GitHub profile
			\raisebox{-1pt}{\faLinkedinSquare} & \href{https://www.linkedin.com/in/agnordby/}{https://www.linkedin.com/in/agnordby/} \\ % LinkedIn profile
			% See fontawesome.pdf in the fonts folder for all icons you can use
		\end{supertabular}
	}
}

%----------------------------------------------------------------------------------------
%	KEY COMPETENCIES
%----------------------------------------------------------------------------------------

\section{Kvalifikasjoner}

\longformdescription{Prosesser}{Systemarkitektur, Scrum, Kanban, prosjektledelse.}

\longformdescription{Utvikling}{Episerver CMS, Episerver Commerce, Episerver Find, C\#, ASP.Net, MVC, Apptus eSales, inRiver PIM, IdentityServer, T-SQL, PL/SQL, JavaScript, JSON, Ajax, NServiceBus, WCF, WebServices, REST, WebAPI, XML, HTML, CSS, jQuery, PowerShell, \LaTeX.}

\longformdescription{Verktøy}{Jira, Confluence, Git, TFS, SubVersion, ReSharper, NewRelic, jMeter, BlazeMeter, NuGet.}


%----------------------------------------------------------------------------------------
%	CERTIFICATIONS
%----------------------------------------------------------------------------------------

\section{Sertifiseringer}

% Example \tableentry{} command to add another line:

%\tableentry{Heading}{Content}{spaceafter}

% All 3 parameters must be supplied but any can be empty if you don't need them
% A "spaceafter" value in the third parameter will add some vertical space -- this is to be used between headings

%------------------------------------------------

\begin{supertabular}{rl} % Start a table with two columns, the table will ensure everything is aligned

	\tableentry{2019}{\textbf{PRINCE2 Practitioner}}{}
	\tableentry{}{\textit{PeopleCert}}{spaceafter}
	
	\tableentry{2019}{\textbf{Certified Scrum Master}}{}
	\tableentry{}{\textit{Scrum Alliance}}{spaceafter}

	\tableentry{2018}{\textbf{PRINCE2 Foundation}}{}
	\tableentry{}{\textit{PeopleCert}}{spaceafter}

	\tableentry{2018}{\textbf{Programming HTML5 with Javascript \& CSS3}}{}
	\tableentry{}{\textit{Microsoft}}{spaceafter}

	\tableentry{2017}{\textbf{Certified Episerver Developer}}{}
	\tableentry{}{\textit{Episerver}}{spaceafter}

	\tableentry{2016}{\textbf{Certified inRiver Developer}}{}
	\tableentry{}{\textit{inRiver}}{spaceafter}



\end{supertabular}



%----------------------------------------------------------------------------------------
%	COURSES
%----------------------------------------------------------------------------------------

\section{Kurs}

% Example \tableentry{} command to add another line:

%\tableentry{Heading}{Content}{spaceafter}

% All 3 parameters must be supplied but any can be empty if you don't need them
% A "spaceafter" value in the third parameter will add some vertical space -- this is to be used between headings

%------------------------------------------------

\begin{supertabular}{rl} % Start a table with two columns, the table will ensure everything is aligned

	\tableentry{2019}{\textbf{PRINCE2 Practitioner}}{}
	\tableentry{}{\textit{Metier OEC}}{spaceafter}
	
	\tableentry{2019}{\textbf{Certified Scrum Master}}{}
	\tableentry{}{\textit{Glasspaper}}{spaceafter}

	\tableentry{2019}{\textbf{Architecture Engineering Workshop}}{}
	\tableentry{}{\textit{Tom Gilb}}{spaceafter}


	
	\tableentry{2018}{\textbf{Arkitekturskolen}}{}
	\tableentry{}{\textit{Sopra Steria}}{spaceafter}




	\tableentry{2016}{\textbf{inRiver Certified Developer}}{}
	\tableentry{}{\textit{inRiver}}{spaceafter}

	\tableentry{2015}{\textbf{Lead Enterprise Architect Programme (LEAP)}}{}
	\tableentry{}{\textit{Microsoft}}{spaceafter}
	
\end{supertabular}




%----------------------------------------------------------------------------------------
%	EDUCATION
%----------------------------------------------------------------------------------------

\section{Utdannelse} 

% Blank \educationentry{} command to add another degree:

%\educationentry{} 
%{} % Degree
%{} % Honours, achievements or distinctions (e.g. first class honours)
%{} % Department
%{} % Institution

% All 5 parameters must be supplied but any can be empty if you don't need them

%------------------------------------------------

\begin{supertabular}{rl} % Start a table with two columns, the table will ensure everything is aligned

	%------------------------------------------------
	
	\educationentry{2003 -- 2004} 
	{Adjunkt med opprykk} % Degree
	{Praktisk-pedagogisk utdanning} % Honours, achievements or distinctions (e.g. first class honours)
	{} % Department
	{Univarsitetet i Oslo} % Institution
	
	\educationentry{1999 -- 2000} 
	{IT-kandidat} % Degree
	{} % Honours, achievements or distinctions (e.g. first class honours)
	{Objekt-orientert programmering.} % Department
	{NæringsAkademiet, Oslo} % Institution
	
	\educationentry{1987 -- 1996} 
	{Candidatus Magisterii} % Degree
	{Gammel grad mellom Bachelor og Master} % Honours, achievements or distinctions (e.g. first class honours)
	{Matematikk, datalingvistikk, japansk.} % Department
	{Universitetet i Oslo} % Institution
	

	%------------------------------------------------

\end{supertabular}


%----------------------------------------------------------------------------------------
%	EMPLOYMENT HISTORY
%----------------------------------------------------------------------------------------

\section{Arbeidserfaring} 

% Blank \educationentry{} command to add another degree:

%\educationentry{} 
%{} % Degree
%{} % Honours, achievements or distinctions (e.g. first class honours)
%{} % Department
%{} % Institution

% All 5 parameters must be supplied but any can be empty if you don't need them

%------------------------------------------------

\begin{supertabular}{rl} % Start a table with two columns, the table will ensure everything is aligned

	\educationentry{05.2019 -- d.d.} 
	{Systemarkitekt \& seniorutvikler} % Degree
	{} % Honours, achievements or distinctions (e.g. first class honours)
	{} % Department
	{Delichon} % Institution

	\educationentry{05.2019 -- d.d.} 
	{Daglig leder} % Degree
	{} % Honours, achievements or distinctions (e.g. first class honours)
	{} % Department
	{Delichon} % Institution

	\educationentry{01.2018 -- 04.2019} 
	{Lead Software Engineer} % Degree
	{} % Honours, achievements or distinctions (e.g. first class honours)
	{} % Department
	{Sopra Steria} % Institution
	
	
	\educationentry{03.2014 -- 08.2017} 
	{Senior Web Developer} % Degree
	{} % Honours, achievements or distinctions (e.g. first class honours)
	{} % Department
	{CGI} % Institution
	

	\educationentry{03.2010 -- 02.2014} 
	{Senior systemkonsulent} % Degree
	{Tarantell fusjonerte med Making Waves} % Honours, achievements or distinctions (e.g. first class honours)
	{} % Department
	{Making Waves / Tarantell} % Institution
	
	\educationentry{08.2008 -- 12.2009} 
	{Konsulent / Systemutvikler} % Degree
	{} % Honours, achievements or distinctions (e.g. first class honours)
	{} % Department
	{Creuna} % Institution
	

	\educationentry{02.2006 -- 07.2008} 
	{Systemutvikler \& Prosjektleder} % Degree
	{InfoFinder fusjonerte med Ocelluz} % Honours, achievements or distinctions (e.g. first class honours)
	{} % Department
	{InfoFinder / Ocelluz} % Institution
	
	
	\educationentry{10.2004 -- 02.2008} 
	{Selvstendig næringsdrivende} % Degree
	{} % Honours, achievements or distinctions (e.g. first class honours)
	{} % Department
	{Anders Nordby Konsulenttjenester} % Institution
	
	\educationentry{05.1995 -- 06.2002} 
	{Konsulent} % Degree
	{} % Honours, achievements or distinctions (e.g. first class honours)
	{} % Department
	{Ajilon} % Institution
	
	\educationentry{08.1994 -- 07.1995} 
	{Lærervikar (årsvikariat)} % Degree
	{} % Honours, achievements or distinctions (e.g. first class honours)
	{} % Department
	{Gran skole, Oslo} % Institution
	

\end{supertabular}


%----------------------------------------------------------------------------------------
%	AWARDS
%----------------------------------------------------------------------------------------

\section{Utmerkelser}

% Example \tableentry{} command to add another line:

%\tableentry{Heading}{Content}{spaceafter}

% All 3 parameters must be supplied but any can be empty if you don't need them
% A "spaceafter" value in the third parameter will add some vertical space -- this is to be used between headings

%------------------------------------------------

\begin{supertabular}{rl} % Start a table with two columns, the table will ensure everything is aligned
	
	\tableentry{2013}{\textbf{Episerver Most Valued Professional (EMVP)}}{}
	\tableentry{}{\textit{Utløpt i 2017}}{spaceafter}
	
\end{supertabular}


%----------------------------------------------------------------------------------------
%	LANGUAGES
%----------------------------------------------------------------------------------------

\section{Språk} 

% Example \tableentry{} command to add another line:

%\tableentry{Heading}{Content}{spaceafter}

% All 3 parameters must be supplied but any can be empty if you don't need them
% A "spaceafter" value in the third parameter will add some vertical space -- this is to be used between headings

%------------------------------------------------

\begin{supertabular}{rl} % Start a table with two columns, the table will ensure everything is aligned
	
	%------------------------------------------------
	
	\tableentry{Utmerket}{Norsk (morsmål), Engelsk}{spaceafter}
	\tableentry{Middels}{Bulgarsk}{spaceafter}
	\tableentry{Nybegynner}{Spansk, Tysk, Fransk, Japansk}{spaceafter}
	\tableentry{For moro skyld}{High Valyrian}{spaceafter}
	
	%------------------------------------------------

\end{supertabular}


%----------------------------------------------------------------------------------------

\end{paracol}

%----------------------------------------------------------------------------------------

\end{document}
